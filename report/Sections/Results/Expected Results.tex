The main objective of this work is to be able to control multiple robots in a cooperative fashion, in order to move a payload in a microgravity environment from a configuration to the other, as well as do a study that is able to find the optimal number of robots that should be used during this movement, since this will allow to for a quantitative analysis of the benefits of adding more robots, since the main advantage more robots will have in a microgravity environment in increasing the force applied to the payload, therefor reducing the time it takes to do the full work. With this in mind at the end of this work we must be able to

\begin{itemize}
    \item Create a path for each robot that ensures the payload movement from one configuration to the other is optimal, while also ensure the robots do not collide with each other during transport.
    \item Create a control system that allow the robots to follow an already planed trajectory, while being flexible enough to allow for changes that might need to be done to avoid unplanned obstacles, and if need be stop the system in place while no other solution is found. 
    \item Compare the simulation environment to the real world environment, in order to evaluate the accuracy of the simulation, since otherwise the results obtained from simulation, where more complex environments can be tests, might not be usable in the real world.

\end{itemize}