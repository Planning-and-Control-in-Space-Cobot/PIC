Cooperative payload transportation is a challenging and emerging field, particularly in a microgravity environment. To address this challenge, the state of the art explored in this work focuses not only on microgravity scenarios but also on cooperative payload transportation in terrestrial environments using Unmanned Aerial Vehicles (UAVs).

Numerous studies have investigated the use of UAVs for payload transportation, employing various connection methods such as ball joints~\cite{tagliabue2019robust,loianno2018localization}, robotic arms~\cite{lee2020visual,ouyang2021control}, and cables~\cite{li2021design,klausen2018cooperative,li2023nonlinear}. These systems present significant challenges requiring advanced control algorithms. For instance,~\cite{dhiman2018cooperative} utilizes a double PID controller to regulate both the drone’s movement and the payload’s damping. However, this approach ignores system dynamics due to the controller's limitations. Improvements are seen in~\cite{gimenez2018multi}, where an inverse kinematic controller operating in null space, meaning we are in space with a higher degree of actuation than degrees of freedom, is employed. This method dynamically controls the system while incorporating redundancy in actuation, enabling the targeting of secondary objectives alongside the primary control goal.

Recent advancements in control theory have introduced Model Predictive Controllers (MPCs) and Nonlinear MPCs (NMPCs)~\cite{li2023nonlinear,9341541}. In~\cite{li2023nonlinear}, an NMPC is used for centralized control of multiple drones. While effective, this centralized approach faces scalability challenges with an increasing number of drones. To overcome this,~\cite{9341541} proposes a distributed MPC, offering a scalable and efficient solution that ensures optimal performance across multiple drones. 

More recently, with the resurgence of artificial intelligence, particularly Deep Reinforcement Learning (DRL), a novel control framework for payload transportation has emerged~\cite{panetsos2022deep,lin2023payload,9345959}. The primary advantage of DRL lies in its ability to learn the full dynamics of the system, including coupled dynamics that are difficult to model traditionally. However, this approach has limitations, such as the time and computational resources required for training. Once trained, the DRL model can be deployed across a wide range of scenarios at a lower operational cost compared to traditional methods.

Payload transportation in microgravity environments remains less explored. Nonetheless, studies on both non-cooperative~\cite{9429795} and cooperative payload transportation~\cite{9438133,9429795,phodapol2024collaborative} have made progress by modeling connections using ropes or rigid links. These studies highlight the increasing complexity of control algorithms. In~\cite{9438133}, a simple PID controller was employed for cooperative payload transportation, allowing basic system collaboration without requiring inter-robot communication. However, this approach lacks optimization. To address these limitations, a centralized NMPC was proposed in the same work, demonstrating effectiveness but with constraints related to trajectory generation time and the computation of optimal actuation inputs. In contrast,~\cite{phodapol2024collaborative} introduced a distributed NMPC approach, achieving faster computation times and offering a more scalable solution.
