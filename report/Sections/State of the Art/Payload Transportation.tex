Cooperative payload transportation is a challenging emerging field, especially in a microgravity environment, to combat this problem the state of the art for this work will cover not only this, but also cooperative payload transportation in an earthly environment with the use of unmanned aerial Vehicles (UAV's)

There are multiple works related to the use of UAV's for payload transportation, in \cite{9341541} a formulation is show that we can achieve with a distributed MPC the same resuls we could expect from a centralized system, while also having the benefits of scalability for a problem with more UAV's working together. 

Related to microgravity environments in~\cite{9438133} a PD (proportional-derivative) controller is proposed for a distributed system capable of transporting a payload on orbit to a desired reference point with little knowledge of not only payload dynamics but also the dynamics of the other robots in the system. In~\cite{9429795} the approach is different, a centralized controller, in this case an NMPC, with full knowledge of the system dynamics as well as the environments is used, it is show that the centralized version of the controller the increase of the complexity of the problem is not negligible, since it causes a noticeable increase in the computational time of the controller, as well as the capability optimal trajectory tracking, this show  a centralized controller cannot allow for a scalable solution. The example of a distributed controller is shown in~\cite{phodapol2024collaborative}, a distributed Linear MPC is used to control a set of robot for microgravity payload through the pulling of a rope attached to both the robot and the payload, a special remark should be make to the fact they were able to linearize the rope behavior with Lagrange multiplier and a system constraint to ensure the rope is always taut.