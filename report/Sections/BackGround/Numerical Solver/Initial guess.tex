Given the nature of the problem at hand, it is crucial to compute the optimal solution for the robot's actuation as quickly as possible. Achieving this requires careful consideration not only of the numerical solver employed but also of the parameters used in the solver, particularly the initial values of the variables at the start of the optimization process. The performance of a solver is highly dependent on its first iteration; therefore, the initial values can significantly impact both the computation time and the likelihood of convergence. This necessitates an explanation of the concepts of cold-start and warm-start initialization.

A \textit{cold-start} refers to the case where the solver initializes the variables either randomly or using default values. In the case of CasADi, as specified in its documentation, the default initial guess is all zeros. This approach leverages no prior knowledge or potential solutions, which can lead to slower convergence or, in some cases, failure to converge. 

In contrast, a \textit{warm-start} involves providing initial values for the variables that are close to the optimal solution. This reduces computation time significantly, which is particularly advantageous when dealing with large-scale optimization problems in real-time systems, as in this work.

When employing the NMPC technique discussed earlier, the use of a warm-start is not only advantageous but straightforward to implement. This is because the optimal control problem solved at time \( t \) provides the optimal control input sequence for all times from \( t \) to \( t+N_c \), where \( N_c \) is the control horizon. Additionally, it computes the expected behavior of the system for the prediction horizon from \( t \) to \( t+N_p \), where \( N_p \) is the prediction horizon. Consequently, the solutions obtained at time \( t \) serve as an excellent initial guess for both the control inputs and the system states at time \( t+1 \). Provided the model is accurate, and the control signal applied at time \( t \) corresponds to the computed solution, the state variables and control sequence from the previous time step offer a good starting point for the new problem. In such cases, the solver only needs to make minor adjustments to account for unmodeled behaviors.

Despite the benefits of warm-starting, it is important to note that the first execution of the solver must use a cold-start, as no prior solution exists. After this initial run, the transition to a warm-start can be made seamlessly.