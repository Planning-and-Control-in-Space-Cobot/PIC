The problem tackled in this work is the problem of payload transportation in a microgravity environment. The environment we simulate in this work is one similar to what is found in the inside of a space station, meaning our payload will be mainly composed by objects similar to boxes, we will need to consider a flexible connection between the object and the robot, since this increases the articulation in between the robot and the payload. This is a complex problem that involves multiple stages. We need to be able to find a path going from one point to the another safely, while also minimizing the time taken in both planing the trajectory and executing it. After a path is found we enter in stage where we must be able to control the robot during the trajectory, this involves not only controlling the position and velocity of the robots, but also the payload, as well as reacting to any changes in the environment, such as the presence of obstacles. Another big problem is the lack of knowledge of the payload, with the use of vision systems, the only information we can obtain about the payload is the size, meaning we will not have information related to the physical parameters (i.e. mass, inertia matrix), this leads to uncertainty and errors that must be dealt with during transportation. Lastly we have the problem of the communication between the robots, since the robots are working in a cooperative manner, they need to be able to communicate with each other, and this communication needs to be fast and reliable, since any delay in the communication will lead to loss of efficiency and in an extreme case scenario, loss of safety during the maneuver.