The problem tackled in this work is the transportation of payloads in a microgravity environment. The environment simulated in this study resembles the interior of a space station, where payloads are primarily box-like objects. A flexible connection between the object and the robot is considered to increase articulation between the robot and the payload. This adds complexity to the problem, which is divided into multiple stages.

First, a safe path must be identified to transport the payload from one point to another. This requires minimizing both the planning and execution time of the trajectory. After finding a path, the robot must be controlled along the trajectory. This involves managing not only the robot's position and velocity but also those of the payload. Additionally, the system must react to environmental changes, such as the presence of obstacles.

Another major challenge is the lack of detailed knowledge about the payload. Using vision systems, only the size of the payload can be determined. Physical parameters such as mass and inertia matrix remain unknown, introducing uncertainties and errors that need to be addressed during transportation.

Lastly, effective communication between robots is critical, as they operate cooperatively. The communication must be fast and reliable. Delays in communication can reduce efficiency and, in extreme cases, compromise safety during the maneuver.
