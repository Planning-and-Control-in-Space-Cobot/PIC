Now that we know what is the main motivation behind this work, we must explain what are the main problems being tackled in this project. The most basic cooperative task a robot can do with is to be able to follow a simple trajectory outlined by a human, in this work we will be working in a higher level of control, where we try to use multiple robots to cooperatively transport a payload from point A to point B, for this to be possible, we must first create a feasible path in between these points for all the robots and the payload, this trajectory is not only a list of coordinate point, but also the acceleration and velocity desired for the robots during the movement. After this we must be able to control the robots during the trajectory, whilst keeping clear of any obstacles, both moving and static that were not known during the trajectory generation phase. 

This task needs to be divided into multiple subtasks to make it feaseble, the first one is being able to generate a feaseble trajectory for the robot's and the payload, and this implies knowing the dynamics of the robots, and the phisical parameters of the payload, such as size and mass. The second task it to be able to control the robots during the trajectory, in our case we will be using holomonic robots, this means we can have decoupled movement in all the degrees of freedom, this mean the main constraint we will have for each robot is phisical limits in acceleration, velocity and force that can be applied. The third task it to be able to detect obstacle at run time, and the the forth task is to be able to either change the original trajectory, or simply taking avoiding action is the new obstacle does not render the previous trajectory unfeaseble.

In this work we will mainly focus in steps 1, 2 and 4, and for step 3 simplification will be made, this does not mean we overlook the importance of obstacle detection in a real world scenario, simply we decided the main focus of the work should be the control and trajectory generation of the robots. The main simplification we will doing for the step 3 will be informing the robot when it comes close to obstacles and giving information such as the distance, direction and size of the robot and unoccupied space, this mean we don't need to use neither a camera nor a lidar for the detection.

For the first step we need an algorithm that is able to create a path for the robot going from point A to point B, but some constraint are needed such as, the path must be feasible, and respect the actuation limits we have for the robot, and we also must be able to generate it in a fast amount of time, otherwise the robot cannot work in a cooperative manner. For the second step, the control loop, we need to create a relieable and safe control loop that is capable of following the trajectory, and since the main objective is for the control loop to be cooperative between multiple robots, we will be working in a environment where redundancy is present, meaning less important control objectives can also be acheived, for example reduce the amount of noise generated by the robot during operation. The point 4 involves also the control loop, since it needs to be both reliable and good in path following, but also flexible enough that it can take avoiding action in real time.  

