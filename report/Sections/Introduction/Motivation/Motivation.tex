Humanity has always been fascinated by space. From a young age, we have gazed at the stars and wondered about the mysteries of the cosmos. This fascination has driven our pursuit of space exploration, leading to groundbreaking discoveries and advancements in our understanding of the universe. The era of man-made objects in space began during the 20th century amidst the Cold War, a time of intense technological competition. By the end of that century, the first modules of the International Space Station (ISS) were launched, marking the beginning of an unprecedented era of human and robotic collaboration in space.

Space robotics has become a key focus of these efforts. Cooperative and autonomous robots are invaluable assets for astronauts, particularly when tasks are either too dangerous or monotonous for human execution~\cite{pedersen2003survey}. These robots extend human capabilities and enhance safety, efficiency, and reliability in space missions.

Since then, numerous projects have explored the use of robots inside space stations. Notable examples include the SPHERES project~\cite{miller2000spheres}, Astrobee~\cite{bualat2018astrobee}, Int-Ball~\cite{mitani2018crew}, and CIMON. Among these, one lesser-known but significant project is the Space Cobot~\cite{RoqueVentura2016spacecobot}, which was developed right here at IST. The Space Cobot will also serve as the model for the robotic system in this project.

As mentioned, the primary goal of these robotic platforms is to work with astronauts. This collaboration offers numerous advantages, including enhanced safety, improved operational efficiency, and reduced costs. The cost factor is particularly critical in the ISS, where research expenses are notably high. For instance, private research onboard can cost as much as \$130,000 per hour, excluding the substantial costs of transporting equipment to orbit, as highlighted on NASA's website\footnote{\url{https://www.nasa.gov/humans-in-space/commercial-and-marketing-pricing-policy/}, last accessed on 21/11/2024}. Thus, increasing the efficiency of work performed on these platforms is essential to making space research more affordable and accessible.
