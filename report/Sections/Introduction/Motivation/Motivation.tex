\par In the last few years, there has been a growing interest in the use of UAV's (unmanned aerial vehicles), this interest comes from the ability  of performing multiple task in a much faster and efficient way since they can be used in a wide range of applications, and execute maneuvers whilst remotely controlled or even autonomously that using a autopilot system capable of Guidance, Navigation and Control (GNC)~\cite{chao2010autopilots}. As we will see in a latter section in details, section~\ref{sec:Background:Space Robots}, this another trend has also been created that is the use of small scale robot inside the Internation Space Station (ISS), these robots have multiple uses, for this work we will be using one of these robots, the Space Cobot~\cite{RoqueVentura2016spacecobot}, developed in the Institute of System and Robotics (ISR) at Instituto Superior Técnico (IST) in Lisbon, Portugal. 

Maintaining and conducting research in the ISS is not only a dangerous task, but also a rather expensive one, with the cost of 130 thousand dollars per hour of research and a limited number hours dedicated to research per year, as shown in  \href{https://www.nasa.gov/humans-in-space/commercial-and-marketing-pricing-policy/}{NASA's website}\footnote{Last accessed at 01 December 2024}, this mean that any small improvement in the efficiency of the research can lead to a large reduction in the cost, and an increase in the number of effective hours of research.

For this reason the creation of space cooperative robots capable of working both autonomously and with humans is a key factor in the future of space exploration, these robots can be used to accelerate multiple tasks, a few examples of use cases are cooperative payload transportation, autonomous inspection of the interior and exterior of the environment, and even autonomous scientific experiments. 