In the year 1955 the space race officially begun, this was the moment where both the United States and USSR anounced they wanted to launch artificial satalite for the the first time in history. During this period of time multiple barriers were broken, such as the first man made satalite orbiting the earth, the first human in space and latter the first human on the moon. And when this race was officialy over the two contries decided to join forces and create the International Space Station (ISS), a space station that is still in orbit today, and now is used by multiple countries as a research lab in orbit.

Maintaining and conducting research in the ISS is not only a dangerous task, but also a rather expensive one, with the cost of 130 thousand dollars per hour of research and a limited number hours dedicated to research per year, as shown in  \href{https://www.nasa.gov/humans-in-space/commercial-and-marketing-pricing-policy/}{NASA's website}\footnote{Last acessed at 01 December 2024}, this mean that any small improvement in the efficiency of the research can lead to a large reduction in the cost, and a larger number of efficitve hours of research.

For this reason the creation of space cooperative robots capable of working both autonomosly and with humans is a key factor in the future of space exploration, this robots can be used to transport payload in large groups, meaning the payload transported can be much heavier than a single robot in that formation, while Maintaining efficiency and safety, can be used to inspect the outside of a space station, saving precious time and increasing safety, since the astronauts don't need to go outside the station, and can even be used to transport astronauts from one module to another. 