\par In recent years, there has been a growing interest in the use of UAVs (Unmanned Aerial Vehicles) due to their versatility and efficiency in performing a wide range of tasks. UAVs offer significant advantages by enabling faster and more effective operations, whether remotely controlled or operating autonomously using autopilot systems capable of Guidance, Navigation, and Control (GNC)~\cite{chao2010autopilots}. As discussed in detail in Section~\ref{sec:Background:Space Robots}, another emerging trend is the deployment of small-scale robots within the International Space Station (ISS). These robots, already in use, serve a variety of purposes and represent a promising direction for space exploration and operations.



Maintaining and conducting research in the ISS is not only a dangerous task, but also a rather expensive one, with the cost of 130 thousand dollars per hour of research and a limited number hours dedicated to research per year, as shown in  \href{https://www.nasa.gov/humans-in-space/commercial-and-marketing-pricing-policy/}{NASA's website}\footnote{Last accessed at 01 December 2024}, this mean that any small improvement in the efficiency of the research can lead to a large reduction in the cost, and an increase in the number of effective hours of research.

For this reason the creation of cooperative space robots capable of working both autonomously and with humans is a key factor in the future of space exploration, these robots can be used to accelerate multiple tasks, a few examples of use cases are payload transportation, autonomous inspection of the interior and exterior of the environment, and even autonomous scientific experiments. 