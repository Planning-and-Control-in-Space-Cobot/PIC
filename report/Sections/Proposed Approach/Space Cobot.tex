Space Cobot \cite{RoqueVentura2016spacecobot} is a holonomic, multirotor modular vehicle designed with modularity for easy maintenance and a wide range of applications. The propulsion system consists of six electric motors arranged to ensure holonomic kinematics.

Each motor in the propulsion module is 4.5 inches in size and operates independently. The configuration of these motors guarantees that the robot maintains holonomic dynamics. It is crucial to consider the bidirectional capability of the motors; although they can exert force in both directions, a slight reduction in thrust occurs when the associated propeller spins in reverse.

For a single propeller \(i\), which is rigidly linked to the robot's body frame (see Figure \ref{fig:Proposed Approach: Space Cobot: Motor graph}), both the thrust (\(F_{i}\)) and the torque (\(M_{i}\)) can be computed using the following equations. Here, \(u_i\) represents the input of the \(i\)-th actuator in rpm (rotations per second), \(K_1\) is the thrust coefficient, \(K_2\) is the torque coefficient, and \(w_i\) is either \(+1\) or \(-1\), depending on whether the propeller rotates clockwise (CW) or counterclockwise (CCW). Note that \(w_i \neq \omega\), where \(\omega\) denotes the robot's angular velocity.

\begin{figure}[H]
    \centering
    \includegraphics[width=0.7\textwidth]{Images/Propposed Aproach/space cobot motor graph.png}
    \caption{Notation used for modeling a single propeller in the Space Cobot robot. Image sourced from \cite{RoqueVentura2016spacecobot}.}
    \label{fig:Proposed Approach: Space Cobot: Motor graph}
\end{figure}

\begin{align}
    F_{i} &= f_i, & f_i &= K_{1} u_i, \\
    M_{i} &= \bar{r_{i}} \times F_{i} - \tau_{i} u_i, & \tau_{i} &= w_i K_{2} u_i
    \label{eq:Proposed Approach: Space Cobot: Motor equations}
\end{align}

The constants \(K_1\) and \(K_2\) are defined as follows (Equation \ref{eq:Proposed Approach: Space Cobot: Motor constants}), where \(\rho\) is the air density, \(D\) is the propeller diameter, and \(C_T\) and \(C_P\) are the thrust and power coefficients, respectively (dimensionless) \cite{mccormick1994aerodynamics}.

\begin{align}
    K_{1} &= \rho D^{4} C_{T}, & K_{2} &= \frac{\rho D^{5} C_{P}}{2\pi}
    \label{eq:Proposed Approach: Space Cobot: Motor constants}
\end{align}

Referring again to Figure \ref{fig:Proposed Approach: Space Cobot: Motor graph}, the position and orientation of each propeller relative to the center of mass (COM) of the robot are given by Equation \ref{eq: Proposed Approach: Space Cobot: Motor position and orientation}:

\begin{align}
    \bar{r_{i}} &= \begin{pmatrix}
        d \cos(\theta_{i}) \\ 
        d \sin(\theta_{i}) \\
        0
    \end{pmatrix}, &
    \hat{u_{i}} &= \begin{pmatrix}
        \sin(\theta_{i}) \sin(\Phi_{i}) \\
        -\cos(\theta_{i}) \sin(\Phi_{i}) \\
        \cos(\Phi_{i})
    \end{pmatrix}
    \label{eq: Proposed Approach: Space Cobot: Motor position and orientation}
\end{align}

Using general robot dynamics, we can stack the forces and moments applied to the robot and relate them to the actuation of all motors. By this logic, we derive a linear relation where:
\[
\begin{pmatrix} F_{i} \\ M_{i}  \end{pmatrix} = \hat{a_{i}} u_{i}
\]
Here, \(\hat{a_i}\) represents the actuation vector of the \(i\)-th motor, which is defined in Equation \ref{eq:Proposed Approach: Space Cobot: Actuation vector}. For the robot to maintain holonomic behavior, the configuration of the actuators must ensure that matrix \(A\) is at least rank 6. The selected configuration is shown in Table \ref{tab:Proposed Aproach: Space Cobot Motor and propeller configuration}.

\begin{equation}
\mathbf{a}_i = 
\begin{pmatrix}
K_1 \sin(\theta_i) \sin(\phi_i) \\
-K_1 \cos(\theta_i) \sin(\phi_i) \\
K_1 \cos(\phi_i) \\
[K_1 d \cos(\phi_i) - w_i K_2 \sin(\theta_i)] \sin(\phi_i) \\
-[K_1 d \cos(\phi_i) - w_i K_2 \sin(\phi_i)] \cos(\theta_i) \\
-K_1 d \sin(\phi_i) - w_i K_2 \cos(\phi_i)
\end{pmatrix}
\label{eq:Proposed Approach: Space Cobot: Actuation vector}
\end{equation}

\begin{table}[H]
\centering
\begin{tabular}{l|llllll}
Propeller (\(i\)) & 0  & 1   & 2   & 3    & 4   & 5   \\ \hline
\(\theta_{i}\)     & 0  & 60  & 120 & 180  & 240 & 300 \\
\(\Phi_{i}\)       & 55 & -55 & 55  & -55  & 55  & -55 \\
\(w_{i}\)          & -1 & 1   & -1  & 1    & -1  & 1   \\ \hline
\end{tabular}
\caption{Configuration of Propeller Placement in Space Cobot}
\label{tab:Proposed Aproach: Space Cobot Motor and propeller configuration}
\end{table}

