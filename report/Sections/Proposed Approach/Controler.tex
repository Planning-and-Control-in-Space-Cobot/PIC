The controller in our system must track the optimal reference previously planned safely. This must be done in real-time, meaning off-the-shelf algorithms will likely struggle, as shown in~\cite{7400956}, where computation time for simpler problems can range from 1 second to 10 and even being unable to find a solution. To ensure optimal actuation while respecting real-time constraints, we employ a distributed NMPC with real-time iterations.

Since the trajectory from the planning phase accounts for the kinodynamics of each robot, it provides a feasible reference. Therefore, our control algorithm primarily focuses on safe tracking. To achieve this, we utilize an NMPC approach with soft constraints designed to prevent contact with unknown objects. Similar to the planning phase, this stage operates distributedly, requiring a consensus phase within the optimization process. This necessity arises from the lack of knowledge about the transported payload, specifically its inertia matrix and mass. Consequently, the reference may not perfectly account for the payload's influence.

Given the reference trajectory, each robot requires individual control to follow it safely. For this, we employ an NMPC with the dynamics described in equation~\ref{eq:Proposed Approach:Motion Model: Multiple Robot Dynamics: Robot Dynamics} and ~\ref{eq:Proposed Approach:Motion Model: Multiple Robot Dynamics: Payload Dynamics}. Our system incorporates both soft and hard constraints. System parameters are enforced as hard constraints to prevent infeasible states within the simulation and control horizon. Soft constraints ensure safety by penalizing proximity to known obstacles and other robots, allowing for minor deviations from the reference.


While a low-order integration method may have lower precision compared to higher-order methods like Runge-Kutta (as noted in~\cite{gros2020linear}), it offers faster computation times, enabling quicker response times. We found that normalizing the quaternion after each integration step within the horizon significantly improves performance, especially for quaternion representations in the state vector. This approach not only yields good results but also outperforms higher-order methods in terms of computational speed.