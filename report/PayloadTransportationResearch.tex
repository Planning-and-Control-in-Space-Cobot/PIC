\documentclass[a4paper,12pt]{article}
\usepackage[utf8]{inputenc}
\usepackage{graphicx}
\usepackage{amsmath}
\usepackage{amsfonts}
\usepackage{amssymb}
\usepackage{hyperref}
\usepackage{geometry}

\usepackage[backend=biber]{biblatex}
\addbibresource{PayloadTransportationResearch.bib}

\geometry{a4paper, margin=1in}
\setcounter{tocdepth}{2}

\begin{document}

% Title page
\begin{titlepage}
	\centering
	\vspace*{1in}
	{\Huge \textbf{Summary of Papers Read in the PIC and Classification}} \\
	\vspace{1.5in}
	{\large \textbf{André Teixeira}} \\
	\vfill
\end{titlepage}

% Table of Contents
\tableofcontents

\clearpage

% Sections
\section{Payload Transportation in Microgravity with Single and Multiple Cooperative Free-Flyer Robots}\label{sec:Correia_Ventura_Payload_Transportation_in_Microgravity}
This paper\cite{correia2021payload}, authored by Rodrigo Ventura and Rui
Correia in 2021, was published in the 2021 IEEE International Conference on
Autonomous Robot Systems and Competitions (ICARSC).

\subsection{Objective}
The main objective of the paper was to create three simulations in Gazebo: a
simple robot moving, a robot moving with a payload, and two robots
cooperatively moving a payload.

To control the robot in Gazebo, a controller node was implemented in ROS,
enabling communication with the robot.

This paper confirms Equations (1) and (6) of the paper we are creating. It also
uses the quaternion derivative equation from Trawny's paper.

To generate the trajectories, the Augmented Lagrangian Trajectory Optimizer
(ALTRO) was used. For the multi-robot simulation, each robot computed the
optimal actuation via NMPC, only applying the one relevant to itself.

\subsection{Trajectory Generation}
A feasible reference trajectory is created using an Optimal Control Problem
(OCP). This trajectory generation is done offline, so compute time is not
critical.

\subsection{Classification of the Paper}
This paper is insightful and offers significant work that can be developed for
my PIC and thesis. Using an OCP for trajectory generation is interesting but
makes the system complex since it cannot run in real-time. Full knowledge of
the environment is required, which limits application in dynamic settings like
the ISS interior.

To extend this work, a dual NMPC approach could be considered: one with a
larger horizon running at a slower frequency for trajectory generation and
another with a smaller horizon at a high frequency (around 50 Hz) for real-time
control.

\textbf{Relevance Rating: 10/10}

\section{Decentralized Trajectory Optimization for a Fleet of Industrial Mobile Robots}\label{sec:Ines_Silva_Thesis_Decentralized_Trajectory_Optimization}
This Master's Thesis, authored by Inês Sofia Baptista Silva, with advisors
Prof. Alberto Vale and Prof. Rodrigo Ventura, was submitted at Instituto
Superior Técnico in November 2021.

\subsection{Objective}
The thesis aimed to develop a decentralized trajectory optimizer for a fleet of
industrial mobile (2D) robots to perform cooperative tasks without collisions
with each other or obstacles.

\subsection{Trajectory}
The trajectory creation was divided into two steps: trajectory generation and
trajectory optimization. The generation was sequential, with each vehicle
generating its trajectory after the previous vehicles.

\subsubsection{Trajectory Generation}
To meet real-time requirements, a fast, feasible trajectory generation
algorithm was needed. A Kinodynamic RRT variant was used for its speed, despite
the lack of guaranteed optimality.

\subsubsection{Trajectory Optimization}
The Kinodynamic RRT provided a random feasible trajectory, which was then
optimized. Initially, the optimization removed unnecessary steps to improve
speed. Later, a new cost function rewarded vehicles that kept to the right side
of corridors, simulating a road for organization and reduced collision risk.

\textbf{Relevance Rating: 9/10}

\section{Nonlinear Model Predictive Control for Trajectory Tracking and Obstacle Avoidance of Industrial Mobile Robots}\label{sec:Francisco_Madeira_Thesis_NMPC_for_reference_tracking_in_industrial_mobile_robots}
This Master's Thesis by Francisco Maria de Lima Raposo Madeira, with Prof.
Alberto Vale and Prof. Rodrigo Ventura, was submitted at Instituto Superior
Técnico in December 2023.

\subsection{Objective}
The main objective was to develop an NMPC for accurately tracking a feasible
trajectory while avoiding static and dynamic obstacles. Trajectory generation
was not a primary focus, but likely relied on similar methods from prior
research.

\subsection{Proposed approach}
	The robot used as constraints in multiple phisical parameters, such as the min and max speed, actuation and angular velocity, with this in mind the trajectory we have to follow can be represented as $trajectory =  \left[\text{state}_{1} \hspace{0.5cm} \text{state}_{2}  \hspace{0.5cm} \dots \hspace{0.5cm} \text{state}_{N} \right]$ where each of the indexed represent the state that the robot should have at time = $t_{i}$, since the number of points in the trajectory wont match the controller frequency we will need to interpolate between each of the point, the error of the interpolation is shown as being irrelevant for the control of the robot. The cost function of the controller can be represented as the one in \ref{eq:Francisco_madeira_thesis_cost_function} this cost function penalised offsets to the reference trajectory as well as reference to the actuation reference, since the reference is out new "zero", otherwise we would not be able to follow the trajectory correctly since a we would try not act, this way we have a term that work as a feed forward of sorts in the NMPC.

	\begin{align}
		l(h + k, \textbf{x}_{u}(k, \textbf{x}_{0}), \textbf{u}(k)) &= A + \lambda B \left[
			\left( X_{curr} - X_{ref} \right)^2 + \left( Y_{curr} - Y_{ref} \right)^2  \right] \nonumber \\
		&\quad + \lambda \left[ \left( {U_{x}}_{curr} - {U_{x}}_{ref} \right)^2 + \left( {U_{y}}_{curr} - {U_{y}}_{ref} \right)^2 
		\right]
		\label{eq:Francisco_madeira_thesis_cost_function}
	\end{align}

\subsection{Trajectory Generation}
The obstacle Avoidance techines explored in this thesis were a scanner (Lidar) in the vehicle and using the maximum distance an obstacle can be as $d_{max} =  N h v_{max}$ where N is the predicion horizon lenght and h is the time step condireted, this for our work is not as usefull since we dont have a lidar in the robot and since our robot as 6DoF and not 3 DoF as the one in this thesis (x, y, and $\theta$ (yaw))

The other method used was an occupancy map where the environment is descretivez in a grid and and each cell is marked with a number that indicates the likelihood of an obstacle being there.
o

\subsubsection{Controller}
The controller used in this thesis, the NMPC, was created using the casadi framework, with the IPOPT solver and run at 5Hz.

\subsection{Usefull material}
Overall looking at this thesis and mine own, we lack cooperative capabilities, since the only objective of this thesis was to follow a trajectory and avoiding obstacles, our system is also more complex as a robot so some adustment would need to be done, also in this thesis the controller that follows the trajectory is non linear, and we it was previosly discussed to have the controlller that follows the trajectory as being a high frequency linear controller, something like an SQ-MPC. The work seems interesing, but i dont really know what can be used for my work.



\section{Predictive Controllers for Load Transportation in Microgravity Environments}
This master thesis\cite{phodapol2023predictive} was written by SUJET PHODAPOL, and was submitted to KTH Royal Institute of Technology in 2023.

\subsection{Objective}
The main objective of this thesis was to developt a Predicitve controller for load Transportation in Microgravity environments, such as a body in orbit, this is actualy the main objective, since the proposed control algorithm have the objective of clearing debris from orbit, to reduce the risk of colosion with other man made, since each of this colision generates more debris, therefor increasing the risk. This a phenomenon called the Kessler syndrome that can make space travel impossible.

\subsection{Proposed approach}
\subsubsection{Vehicle Definition}
The first step in the development of the final control algorithm is the definition of the robot that will be used for testing and its dynamics, since the robot is for use in orbit then the robot will need to use thruster based propulsion system, this comes as a consequence of the lack of friciton, meaning the only way for a robot to move is to expell a small mass at a large speed, meaning for each mission and for each robot the amount of energy (fuel) available is limited. The author then explains the mechanical systems of the robots, and the mathematical equation for the effect of each thruster in the robot force and thrust vector. 

After this a model for a n vehicle towing a load is created, explaining what each thruster in each vehicle will cause the full system (n vehicles and load) to do.

\subsubsection{Controller Design}
Using the equation presented in the previous chapter a simple NMPC can be created, the only problem with this is the high computational load required for solving the optimization problem. To solve this problem a linearized version of the system is presented, this linearization work by introducing lagrange multipliers, this optimizatin thecnique is applied to the computation of the cable tensions, this means we must not introduce a constraint to make sure the cable is always tensioned, since otherwise the movement of the load will not be impacted by the movement of the robot, this cabe be done by simply saying the force applied by a robot must be greater or equat to zero. Another way to decrease the computation power required is to convert the system dfrom a centralized MPC to decentralized one, this is one where multiple robots compute the best action for them self according to soome consensus algorithm, until a decision is done, when a consensus is reached the robots apply the action to the system. This generated uncertainty, speacily since we cannot know the position of the other robots, neither the force being apllied by them, so we must use a estimation for this 'outside force' in all the optimization problems.


\subsection{Results}
The result shown in this work seem promissing, showing fast computational time for both the centralized and the decentralized control with faster time, as expected in the decentralized controller, the results also show that the centralized controller seems to be faster reaching the desired trajectory for the system, this means a certain trade off must be made in between acceptable time for the controller and the best resulst.

\subsection{Relevance to work}
This seem like a highly important work for my own thesis, since the topic are quite similiar, a multiplle methods are used in both works. The main difference is that this work it to be used in the exterior environment of a space station, and my own will be for the inside. Good ideia for both the conntroller as well as the simulation thecniques used can be taken. Even tho for my own work we have currently decided to consider the joint between payload and robot as a rigid body, this work show that a cable can be used with not much complication.


\clearpage

\printbibliography
\end{document}
