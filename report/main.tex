\documentclass[a4paper,12pt]{article}
\usepackage[utf8]{inputenc}
\usepackage{graphicx}
\usepackage{amsmath}
\usepackage{amsfonts}
\usepackage{amssymb}
\usepackage{hyperref}
\usepackage{geometry}
\geometry{a4paper, margin=1in}
\setcounter{tocdepth}{2}

\begin{document}

% Title page
\begin{titlepage}
    \centering
    \vspace*{1in}
    {\Huge \textbf{Resume of Papers Read in the PIC and Classification}} \\
    \vspace{1.5in}
    {\large \textbf{André Teixeira}} \\
    \vfill
\end{titlepage}

% Table of Contents
\tableofcontents

\clearpage

% sections
\section{Payload Transportation in Microgravity with Single and Multiple Cooperative Free-Flyer Robots}

    \subsection{Objective}
    The main objective in the paper was to create in Gazebo three simulations: one of a simple robot moving, one of a robot moving with a payload, and the last one with two robots cooperatively moving a payload.

    To move the robot inside the simulation in Gazebo, a controller node was implemented in ROS, and this communicated with the robot.

    This paper confirms Equation (1) and Equation (6) of the paper we are creating. They also use the quaternion derivative equation that is found in Trawny's paper.

    To generate the trajectories of the robot, the Augmented Lagrangian Trajectory Optimizer (ALTRO) was used. For the multi-robot simulation, each robot computed the optimal actuation via NMPC but only used the one that is respective to itself.

    \subsection{Trajectory Generation}
    A feasible reference trajectory is created for the controller using an Optimal Control Problem (OCP). The trajectory generation is done offline (before the movement), meaning the compute time is not important.
    
    \subsection{Classification of the Paper}
    This paper, overall, is really interesting and has so much work that can be used and improved upon to develop something for my PIC and thesis. The whole concept of using an OCP to generate a new trajectory is interesting but results in a much more complex system since this process is slow and cannot be done in real time. Full knowledge of the environment is needed. This can be useful to some extent, but when working in an environment that is not static (such as the interior of the ISS), it will not work since we have to react to the movements of the astronauts in real time; otherwise, collisions will eventually happen.

    To expand on this work, something like a double NMPC could be developed, with one having a larger horizon running at a much slower frequency for generating the trajectory and a second one with a smaller horizon running at a high frequency (around 50Hz would be ideal, I would say) to control the robot in real time and ensure the trajectory is followed as closely as possible.

    \textbf{Classification of relevance of work: 10/10}

\clearpage

\section{Nonlinear Model Predictive Control with Actuator Constraints for Multi-Rotor Aerial Vehicles}

    \subsection{Abstract}
    This paper proposes, tests, and validates an online Nonlinear Model Predictive Control (NMPC) method applied to multi-rotor aerial systems with arbitrarily positioned and oriented rotors. The proposed control framework challenges common simplifications such as: (1) linear model approximation, (2) cascaded control paradigms that decouple translational and rotational dynamics, and (3) conservative input/state constraints. Instead, it addresses the problem of local reference trajectory planning and vehicle dynamics stabilization using derivatives of the forces generated by the vehicle. A novel actuator modeling approach avoids conservative input and state saturations common in traditional cascaded approaches. The effectiveness of the method is validated through real-time simulations and experiments on various multi-rotor systems, including under-actuated and fully actuated configurations.

    \subsection{Introduction}
    The use of Multi-Rotor Aerial Vehicles (MRAVs) has increased in recent years due to advances in hardware and software algorithms. These systems are preferred for their versatility, vertical take-off and landing capabilities, agility, and relatively compact structures. However, conventional multi-rotor platforms with fixed thrust directions are less efficient in complex physical interaction tasks. Fully actuated systems, which can independently control position and orientation, offer greater potential but require advanced control strategies.

    Traditional controllers for MRAVs rely on linearized models and classical Proportional-Integral-Derivative (PID) strategies, limiting achievable performance. This paper aims to overcome these limitations by introducing an NMPC strategy that leverages the dynamic model of the MRAV, incorporates actuator constraints, and optimizes performance over a finite time horizon.

    \subsection{Proposed NMPC Approach}
    The core contribution of this paper is the development of an NMPC framework that considers both the translational and rotational dynamics of the MRAV without the simplifications of traditional cascaded control paradigms. The key features of the proposed approach include:

    \begin{itemize}
        \item \textbf{Full-order nonlinear model}: The control algorithm utilizes a detailed model of the MRAV, accounting for arbitrary actuator configurations.
        \item \textbf{Actuator modeling and constraints}: A novel approach to modeling actuators is presented, treating the derivatives of generated forces as control inputs. This avoids conservative constraints typical in cascaded approaches.
        \item \textbf{Real-Time Iteration (RTI) scheme}: The control algorithm is implemented using a state-of-the-art RTI scheme with partial sensitivity updates, allowing efficient online computation.
    \end{itemize}

    \subsubsection{Mathematical Modeling}
    The MRAV is modeled as a rigid body with mass \( m \) and inertia matrix \( J \). The actuators are represented as force-generating units with specific positions and orientations relative to the vehicle's body frame. The NMPC framework formulates a constrained optimization problem to find the optimal control inputs, minimizing a cost function that penalizes deviations from a desired reference trajectory.

    The control framework accounts for various physical constraints, including maximum and minimum torques, speed limits, and actuator force limits. The formulation avoids introducing fictitious constraints on the state variables, resulting in improved closed-loop performance.

    \subsection{Implementation and Validation}
    The NMPC algorithm is implemented using the Real-Time Iteration (RTI) scheme with a fixed-step 4th-order Runge-Kutta integrator. The control algorithm runs at a frequency of 200 Hz, ensuring real-time applicability. The optimization problem is solved at each time step using the qpOASES solver, which employs an active-set method with warm-start capabilities.

    The performance of the proposed NMPC framework is validated through a series of experiments on different multi-rotor platforms, including:
    \begin{itemize}
        \item \textbf{Under-actuated quadrotor}: A conventional quadrotor with four fixed rotors.
        \item \textbf{Fully actuated hexarotor}: A hexarotor platform with non-collinear, tilted rotors, capable of generating forces in multiple directions without reorienting the body frame.
        \item \textbf{MRAV with rotor failure}: A platform with one rotor intentionally disabled to test the robustness of the control algorithm.
    \end{itemize}

    \subsubsection{Experimental Results}
    The experiments demonstrate the ability of the NMPC framework to maintain stability and achieve high performance in various scenarios. Key findings include:
    \begin{itemize}
        \item The NMPC approach successfully handles actuator constraints, achieving precise trajectory tracking even under aggressive maneuvers.
        \item The framework is capable of stabilizing the MRAV in the presence of rotor failures, highlighting its robustness.
        \item The proposed algorithm outperforms traditional cascaded control approaches by avoiding fictitious state constraints and fully exploiting the dynamic capabilities of the platform.
    \end{itemize}

    \subsection{Conclusion and Future Work}
    This paper presents a novel NMPC framework for multi-rotor aerial vehicles that effectively handles actuator constraints and avoids the limitations of traditional cascaded control approaches. The proposed method leverages a full-order nonlinear model and incorporates realistic actuator dynamics, resulting in improved performance and robustness.

    Future research directions include extending the framework to account for more complex aerodynamic effects, improving computational efficiency for real-time implementation on embedded systems, and exploring the application of the proposed approach to swarms of MRAVs.

\clearpage

\section{3DVFH+: Real-Time Three-Dimensional Obstacle Avoidance Using an Octomap}  
    \subsection{Introduction}
    Autonomous robots operating in three-dimensional environments, such as aerial and underwater robots, need effective obstacle avoidance strategies. Traditional techniques involve creating 2D maps from 3D data or using multiple altitude levels, but these approaches are insufficient for robots moving in 3D. This paper proposes 3DVFH+, an algorithm that extends the Vector Field Histogram (VFH+) method by utilizing the octomap framework to represent a 3D environment. This new method addresses the limitations of existing 2D-based techniques by generating real-time obstacle avoidance paths based on the octomap.
    
    \subsection{Octomap Framework}
    The octomap is a 3D occupancy grid framework based on the octree structure. It represents an environment using cubic volumes, or "voxels," arranged hierarchically. Each voxel can contain eight child voxels, which are either occupied or free. If a voxel contains all eight occupied child voxels, the parent voxel is also marked as occupied. The probabilistic representation of the environment in the octomap accounts for sensor noise and uncertainties. The octomap efficiently stores and manages the 3D environment data, making it ideal for real-time obstacle avoidance.
    
    \subsection{3DVFH+ Algorithm}
    The 3DVFH+ algorithm extends the 2D VFH+ algorithm to three dimensions using five main stages:
    
    \subsubsection{Stage 1: Octomap Exploration}
    The algorithm first identifies voxels within a bounding box centered around the robot's position, termed as the Vehicle Center Point (VCP). Only voxels that fall within the bounding box are considered active and further explored. This limits the computational complexity by ignoring irrelevant regions of the octomap.
    
    \subsubsection{Stage 2: 2D Primary Polar Histogram}
    Using the identified active voxels, a 2D polar histogram is created. The histogram is based on two angles: azimuth and elevation, which represent the relative position of the active voxels with respect to the VCP. Weights are assigned to each voxel based on its distance from the VCP and occupancy probability, indicating the potential obstacle's influence.
    
    \subsubsection{Stage 3: Physical Characteristics Integration}
    In this stage, the physical characteristics of the robot, such as size and turning speed, are considered. The algorithm calculates the turning trajectory and checks if any active voxel lies on the turning circle. This ensures that the planned path accounts for the robot's motion constraints.
    
    \subsubsection{Stage 4: 2D Binary Polar Histogram}
    The algorithm creates a binary version of the 2D primary polar histogram by comparing voxel weights against two thresholds. If a voxel’s weight exceeds the high threshold, it is marked as an obstacle; if it is below the low threshold, it is marked as free. This binary histogram reduces noise and false detections from the sensor data.
    
    \subsubsection{Stage 5: Path Detection and Selection}
    The final stage involves detecting and selecting the optimal path using a moving window that searches for continuous free paths in the 2D binary polar histogram. Path weights are calculated based on the distance to the target direction, the robot’s current heading, and the previous direction. The path with the lowest weight is chosen as the robot's next motion.
    
    \subsection{Results}
    The 3DVFH+ algorithm was implemented using the Robot Operating System (ROS) and tested in a Gazebo simulation environment with a quad-copter robot. The robot successfully navigated through a 3D environment, avoiding obstacles and adapting its path in real-time. The results indicate that the algorithm can generate a new robot motion in an average time of 326 $\mu$s, which corresponds to an update rate of 3061 times per second.
    
    \subsection{Conclusion}
    The 3DVFH+ algorithm is a reliable and real-time solution for obstacle avoidance in three-dimensional environments. By combining the octomap framework with the enhanced VFH+ technique, the algorithm effectively detects and avoids obstacles in real-time. The approach only considers obstacles within a specific range of the robot, reducing computational overhead and increasing efficiency. Future improvements include automating the configuration of thresholds and dynamically adjusting them based on the robot's speed.
    
    \subsection{classification}
    
    This papers in focused in creating a path for a robot capable of moving of movement in XYZ (3D) in space, with the use of a depth camera the algorithm can compute a path taking into consideration the robot characteristics from size to turning speed to create a feasilble path in the world.

    This work is an important one, but falls outside the scope of the current desired architecute for the control pipeline for space cobot, since the object if to generate the path in a second NMPC running in a slower frequency, althoug currently there has not been a discussion related to how obstacles will be discovered from the sensors the robot has.

    Classification of relevance of work: 2/10

\section{A Swarm Control Model Based on Individual
Information Interaction}
    \subsection{Abstract}
        This paper proposes a new control model for swarm-robot systems to address challenges in obstacle avoidance and formation maintenance in complex environments. The model combines a virtual leader method and an artificial potential field (APF) technique to prevent collisions and maintain cohesion. The approach uses simple behavior rules and information interactions to achieve effective swarm behavior. Experiments demonstrate the model's effectiveness in both obstacle-free and mixed static-dynamic obstacle environments.

    \subsection{Introduction}
        Swarm-robot systems emulate natural swarm behaviors by following simple interaction rules among individuals. Traditional methods like APF, virtual structures (VS), and leader-follower (LF) approaches are common but have limitations in complex environments. This paper introduces a novel model combining a virtual leader strategy with APF to address these limitations.

    \subsection{Proposed Model}
        The proposed model consists of:
        \begin{itemize}
            \item \textbf{Virtual Leader Model}: Guides the swarm’s overall movement by generating a concentration field inspired by gene regulation networks.
            \item \textbf{APF Model}: Prevents collisions and maintains the formation by combining velocities influenced by the virtual leader, neighboring robots, and obstacles.
        \end{itemize}

    \subsection{Experiments}
        The model was tested in two scenarios: a no-obstacle environment and an environment with mixed static and dynamic obstacles. A metric called the "degree of crowding" was used to evaluate performance. Results indicate that the model effectively maintains swarm formation and avoids obstacles autonomously.

    \subsection{Conclusion}
        The proposed model successfully combines the virtual leader and APF methods to achieve cohesive swarm behaviors in complex environments. Future work includes refining parameters and improving adaptability.


  
    \subsection{Classiciation}
        Overall the swarm like control that is proposed in this papers is somthing rather simplistic and not very useful for the desired use case of the Space Cobot project, the control algorithm is not cooperative or smart in an way shape or form, the robots simply avoid colision with the other robots and object, but in the end, they are not working together to acheive a common goal such as moving a payload from one place to another.

    Classification of relevance of work: 1/10

\section{From linear to nonlinear MPC: bridging the gap via
the real-time iteration}

    \subsection{Introduction}
    Linear MPC is widely employed due to its computational efficiency and effectiveness in dealing with multi-input, multi-output linear systems. However, traditional linear MPC methods struggle with nonlinear dynamics and constraints. Nonlinear MPC offers an alternative, but it has historically faced challenges due to computational demands. The RTI-based NMPC method introduced in this paper bridges this gap by leveraging a specialized iteration scheme, allowing NMPC to achieve near-linear computational efficiency.
    
    \subsection{Linear and Nonlinear MPC Formulation}
    The paper starts by outlining the linear MPC formulation, solving an optimal control problem at each discrete time step. The authors describe how linear MPC approximates nonlinear systems by using linearization techniques. This contrasts with the nonlinear formulation, which directly addresses nonlinear dynamics and constraints through sequential quadratic programming (SQP).
    
    \subsubsection{Real-Time Iteration Scheme}
    The RTI scheme is introduced as a method that optimally addresses NMPC’s computational complexity. RTI performs a single Newton step per time interval, using the previous time-step solution as a starting point. By doing so, it capitalizes on the similarity between consecutive control problems and reduces convergence times significantly.
    
    \subsection{Comparison and Analysis}
    The authors present a detailed analysis comparing the open-loop and closed-loop behavior of linear MPC, RTI-based NMPC, and fully converged NMPC. The RTI method closely approximates the fully converged NMPC solution while maintaining computational efficiency. They demonstrate that under certain conditions, RTI delivers solutions that are only marginally different from those of linear MPC.
    
    \subsection{Numerical Methods and Implementation}
    To further illustrate the benefits of RTI-based NMPC, the paper discusses its implementation, focusing on discretization techniques and sensitivity propagation. The authors highlight several numerical integration methods, including Runge-Kutta, and discuss their impact on the accuracy of the discrete-time nonlinear prediction model.
    
    \subsection{Results and Conclusion}
    The results indicate that RTI-based NMPC provides a reliable, near-linear solution for control problems involving nonlinear dynamics. This method achieves a balance between computational efficiency and control accuracy. The authors conclude by noting that RTI offers a practical and scalable approach to real-time control applications, bridging the gap between linear and nonlinear MPC.

    \subsection {Classification}
    
        The work developed in this papers is a rather theoritical explaining how to go from MPC to NMPC and their similiarities and difereneces, could be a good paper to explain what was done for developing the control strategy of the Space cobot, and even to try and implement the some strategies to get better computational efficiency.

        Classification of relevance of work: 5/10
    
\section{Experimental Evaluation of Model Predictive Control and Inverse Dynamics Control for Spacecraft Proximity and Docking Maneuvers}


    \subsection{Abstract}
        This paper presents an experimental study comparing the performance of two guidance and control algorithms for spacecraft docking maneuvers in a multi-constrained environment. The evaluated methods are Model Predictive Control (MPC) and Inverse Dynamics in the Virtual Domain (IDVD). An LQ-based MPC approach using quadratic programming (QP) and an IDVD algorithm with nonlinear programming (NLP) are implemented and tested in a ground-based hardware-in-the-loop test bed. The study focuses on their handling of complex constraints such as keep-out zones, entry cones, and chaser vehicle actuation limits.

    \subsection{Introduction}
        Rendezvous and Proximity Operations (RPO) are vital for various space missions, such as docking, satellite servicing, and debris removal. Developing reliable and efficient guidance and control algorithms for these operations is essential. This paper focuses on two advanced approaches—Linear-Quadratic Model Predictive Control (LQ-MPC) and Inverse Dynamics in the Virtual Domain (IDVD). The comparison evaluates each approach's constraint-handling capabilities and efficiency in a docking scenario.

    \subsection{Problem Formulation}
        The experimental scenario involves a spacecraft docking maneuver with three critical constraints: a keep-out zone, an entry cone, and maximum allowable force. The test was performed in the POSEIDYN air-bearing test bed, a ground-based setup that simulates zero-gravity conditions on a planar granite surface. The Linear-Quadratic MPC approach handles linear constraints efficiently but must linearize nonlinear constraints, which can lead to sub-optimality. The IDVD approach directly handles nonlinear constraints but involves solving a complex nonlinear optimization problem.

    \subsection{LQ-MPC Approach}
        The LQ-MPC method is based on a receding horizon strategy that solves a Quadratic Programming (QP) problem at each time step. The optimization considers linearized constraints and generates a series of control inputs over a fixed time horizon. The implementation efficiently handles linear constraints, offering guaranteed convergence and robustness. The main drawback is the requirement to approximate nonlinear constraints.

    \subsection{IDVD Approach}
        The IDVD method optimizes the entire docking maneuver by parametrizing the trajectory using polynomial functions of virtual time. It solves a nonlinear programming (NLP) problem to determine the optimal trajectory and control inputs. While the IDVD approach offers flexibility in handling nonlinear constraints, it requires significantly more computational resources and lacks guaranteed convergence.

    \subsection{Experimental Setup and Implementation}
        The experiments were conducted using Floating Spacecraft Simulators (FSS) in the POSEIDYN test bed. Each FSS floats over a 4-by-4 m granite table using air bearings, simulating two degrees of translational motion and one rotational degree of freedom. The onboard cold-gas thrusters are controlled by an embedded system executing the MPC and IDVD algorithms in real-time. Absolute navigation data was provided by a VICON motion capture system augmented with onboard sensors.

    \subsection{Results}
        The experiments showed that both the LQ-MPC and IDVD approaches successfully guided the chaser spacecraft to the docking target while avoiding the keep-out zone. The LQ-MPC exhibited robust performance but required more control effort due to the linearization of constraints. In contrast, the IDVD approach produced more propellant-efficient trajectories but was more susceptible to computational issues, particularly near constraint boundaries.

    \subsection{Discussion}
        The results indicate that the IDVD method yields optimal trajectories in terms of control effort due to its ability to directly handle nonlinear constraints. However, this advantage comes at the cost of increased computational complexity and the risk of convergence failure. The LQ-MPC approach, on the other hand, offers guaranteed convergence and better robustness, albeit with a higher control effort due to constraint linearization. The choice between these methods depends on the specific mission requirements, such as computational capabilities and safety considerations.

        \subsection{Conclusion}
        This study experimentally demonstrates the feasibility of both LQ-MPC and IDVD approaches for spacecraft docking maneuvers in a hardware-in-the-loop setup. The LQ-MPC provides a robust and computationally efficient solution, while the IDVD method offers more optimal trajectories but at a higher computational cost. Future work may involve further optimizing these methods for real-time implementation on flight hardware.

\section{Nonlinear Model Predictive Control for Spacecraft Rendezvous and Docking with a Rotating Target}
        \subsection{Abstract}
        This paper presents a Nonlinear Model Predictive Control (NMPC) strategy for spacecraft rendezvous and docking (RVD) with a rotating target platform. The NMPC approach addresses collision-free docking while handling complex constraints, such as thrust limits, collision avoidance, and spacecraft positioning within an entry cone. A switching algorithm is introduced to dynamically adjust constraints, ensuring safe docking. The NMPC controller's real-time performance is tested using both simulations and experiments on an air-bearing test bed.
        
        \subsection{Introduction}
        Spacecraft rendezvous and docking with rotating or tumbling platforms are critical for missions such as satellite recovery and debris removal. In this context, autonomous docking maneuvers must handle both state and control constraints in real-time. This paper proposes an NMPC strategy capable of solving the control problem while enforcing constraints on the chaser spacecraft's position, thrust, and collision avoidance during docking.
        
        Previous research includes the development of glideslope-based guidance methods for tumbling target docking and the use of optimal control for energy-efficient maneuvers. However, these approaches often fail to handle complex, real-time constraints required in practical scenarios. This paper introduces a real-time NMPC method that handles these constraints effectively.
        
        \subsection{Mission Design and Docking Strategy}
        The mission considered involves docking a chaser spacecraft with a rotating target platform. The maneuver consists of two phases: free-flying and final docking approach. During the free-flying phase, the chaser positions itself at the entry of the docking cone corridor, which guides it towards the docking port. The final docking phase uses a dynamically reconfigurable collision avoidance constraint to ensure precise attitude alignment and safe docking.
        
        \subsection{Nonlinear Model Predictive Controller Design}
        The NMPC controller is designed to minimize a cost function that incorporates state and control penalties while adhering to the spacecraft's dynamics. The spacecraft's dynamics are modeled using the Clohessy-Wiltshire-Hill (CWH) equations, which reduce to double integrator dynamics for short-duration maneuvers. The controller considers constraints on thrust, approaching cone boundaries, and collision avoidance.
        
        \subsubsection{Constraints}
        \begin{itemize}
            \item \textbf{Thrust Constraints:} The chaser spacecraft's thrust is constrained to a maximum allowable force to reflect real-world limitations.
            \item \textbf{Approaching Cone Constraints:} These ensure the chaser remains within a safe corridor during docking.
            \item \textbf{Collision Avoidance Constraints:} A keep-out zone surrounds the target, preventing collisions, while the radius of this zone decreases as the chaser approaches the target.
        \end{itemize}
        
        The NMPC controller optimizes the control inputs by solving a nonlinear optimization problem using the IPOPT solver, ensuring real-time performance.
        
        \subsection{Simulation Results}
        The NMPC controller was tested in simulations using MATLAB/Simulink. Three test cases with different angular velocities for the target were evaluated. The results demonstrated that the controller successfully guided the chaser to dock with the target while avoiding collisions. The controller's average computation time was 0.02 seconds, sufficient for real-time applications.
        
        \subsection{Experimental Validation}
        The proposed NMPC strategy was experimentally validated using an air-bearing test bed, where two spacecraft simulators represented the chaser and target platforms. The experiment involved docking with a rotating target at 0.5 degrees per second. The results closely matched the simulation, with the chaser successfully docking within 107 seconds and exhibiting robust performance under real-time constraints.
        
        \subsection{Conclusion}
        This paper presents a successful implementation of an NMPC strategy for spacecraft rendezvous and docking with a rotating target. The strategy effectively handles dynamic constraints and ensures safe, collision-free docking. Both simulation and experimental results validate the real-time capability of the NMPC approach, making it a promising solution for future autonomous space missions.

\end{document}